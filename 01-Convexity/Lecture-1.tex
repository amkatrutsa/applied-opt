\documentclass[12pt]{beamer}
\usepackage{../latex-sty/mypres}
\usepackage[utf8]{inputenc}
\usepackage[russian]{babel}
\usepackage[T2A]{fontenc}

\expandafter\def\expandafter\insertshorttitle\expandafter{%
  \insertshorttitle\hfill%
  \insertframenumber\,/\,\inserttotalframenumber}
  
\title[Введение]{Введение. Выпуклые множества и выпуклые функции}
\author{Александр Катруца}
\institute{Московский физико-технический институт,\\
Факультет Управления и Прикладной Математики} 
\date{\today}

\begin{document}
\begin{frame}
\titlepage
\end{frame}

\begin{frame}{Зачем этот курс?}
\begin{itemize}
\item Формализация задачи выбора элемента из множества
\item Обоснование правильности принятия решения
\item Разнообразные приложения:
\begin{itemize}
\item машинное обучение: классификация, кластеризация, регрессия
\item молекулярное моделирование
\item анализ рисков
\item выбор активов (portfolio optimization)
\item оптимальное управление
\item обработка сигналов
\item оценка параметров в статистике 
\item и другие\footnote{\url{http://www.cvxpy.org/en/latest/examples/index.html}}
\end{itemize}
\end{itemize}
\end{frame}

\begin{frame}{О чём этот курс?}

\begin{itemize}
\item Основы выпуклого анализа
\item Теория двойственности
\item Условия оптимальности
\item Методы безусловной минимизации первого и второго порядка 
\item Методы условной оптимизации
\item Оптимальные методы
\item ...
\end{itemize}
\end{frame}

\begin{frame}{План на семестр}
\begin{itemize}
\item Семинар-лекция раз в неделю
\item Промежуточная контрольная в середине семестра
\item Итоговая контрольная в конце семестра
\item Домашние задания в течение семестра
\end{itemize}
\end{frame}

\begin{frame}{Предварительные навыки}
\begin{itemize}
\item Линейная алгебра
\item Математический анализ
\item Программирование: Python (NumPy, SciPy, CVXPY) или MATLAB
\item Элементы вычислительной математики
\end{itemize}
\end{frame}

\begin{frame}{Методология}
Основные этапы использования методов оптимизации при решении реальных задач:
\begin{enumerate}
\item Определение целевой функции
\item Определение допустимого множества решений
\item Постановка и анализ оптимизационной задачи
\item Выбор наилучшего алгоритма для решения поставленной задачи
\item Реализация алгоритма и проверка его корректности
\end{enumerate}

\end{frame}

\begin{frame}{Постановка задачи}
\begin{equation*}
\begin{split}
&\min\limits_{\bx \in X} f_0(\bx)\\
\text{s.t. } & f_i(\bx) = 0, \; i = 1,\ldots,p\\
& f_j(\bx) \leq 0, \; j = n+1,\ldots,m,
\end{split}
\end{equation*}
\begin{itemize}
\item $\bx \in \bbR^n$~--- искомый вектор
\item $f_0(\bx): \bbR^n \rightarrow \bbR$~--- целевая функция
\item $f_k(\bx): \bbR^n \rightarrow \bbR$~--- функции ограничений
\end{itemize}
Пример: выбор объектов для вложения денег и определение в какой объект сколько вкладывать
\begin{itemize}
\item $\bx$~--- размер инвестиций в каждый актив
\item $f_0$~--- суммарный риск или вариация прибыли
\item $f_k$~--- бюджетные ограничения, min/max вложения в актив, минимально допустимая прибыль
\end{itemize}

\end{frame}

\begin{frame}{Как решать?}
В общем случае:
\begin{itemize}
\item NP-полные
\item {\small рандомизированные алгоритмы: время vs стабильность}
\end{itemize}

{\small НО определённые классы задач могут быть решены быстро!}

\begin{itemize}
\item Линейное программирование
\item Метод наименьших квадратов
\item Малоранговое приближение порядка $k$
\item Выпуклая оптимизация
\end{itemize}
\end{frame}

\begin{frame}{Линейное программирование}
\begin{equation}
\begin{split}
&\min\limits_{\bx \in \bbR^n} \bc^{\T}\bx\\
\text{s.t. } & \ba^{\T}_i \bx \leq c_i, \; i = 1,\ldots, m
\end{split}
\label{eq::lin_prog}
\end{equation}
\begin{itemize}
\item нет аналитического решения
\item существуют эффективные алгоритмы
\item разработанная технология
\item симплекс-метод для решения задачи~\eqref{eq::lin_prog} входит в Top-10 алгоритмов XX века\footnote{\url{https://www.siam.org/pdf/news/637.pdf}}
\end{itemize}
\end{frame}

\begin{frame}{Метод наименьших квадратов}
\begin{equation}
\min\limits_{\bx \in \bbR^n} \|\bA\bx - \mathbf{b} \|^2_2,
\label{eq::least_sq}
\end{equation}
где $\bA \in \bbR^{m \times n}$ и $\mathbf{b} \in \bbR^m$.
\begin{itemize}
\item имеет аналитическое решение: $\bx^* = (\bA^{\T}\bA)^{-1}\bA^{\T}\mathbf{b}$
\item существуют эффективные алгоритмы
\item разработанная технология
\item имеет статистическую интерпретацию
\end{itemize}
\end{frame}

\begin{frame}{Малоранговое приближение ранга $k$}

\begin{equation}
\begin{split}
&\min\limits_{\bX \in \bbR^{m \times n}} \|\bA - \bX \|_F \\
\text{s.t. } & \text{rank}(\bX) \leq k
\end{split}
\label{eq::lowrank}
\end{equation}

\begin{Theorem}[Eckart–Young, 1993]
Пусть $\bA = \bU\bSigma\bV^{\T}$~--- сингулярное разложение матрицы $\bA$, где $\bU = [\bU_k, \bU_{r-k}] \in \bbR^{m \times r}$, $\bSigma = \diag{\sigma_1, \ldots, \sigma_k, \ldots, \sigma_r}$, $\bV = [\bV_k, \bV_{r - k}] \in \bbR^{n \times r}$ и $r = \text{rank}(\bA)$.  Тогда решение задачи~\eqref{eq::lowrank} можно записать в виде:
\[
\bX = \hat{\bU} \hat{\bSigma} \hat{\bV}^{\T},
\]
где $\hat{\bU} \in \bbR^{m \times k}$, $\hat{\bSigma} = \diag{\sigma_1, \ldots, \sigma_k}$, $\hat{\bV} \in \bbR^{n \times k}$.
\end{Theorem}
\small Алгоритм вычисления сингулярного разложения и быстрый, и устойчивый.
\end{frame}

\begin{frame}{Выпуклая оптимизация}
\begin{equation}
\begin{split}
&\min\limits_{\bx \in \bbR^n} f_0(\bx)\\
\text{s.t. } & f_i(\bx) \leq b_i, \; i = 1,\ldots, m
\end{split}
\label{eq::conv_opt}
\end{equation}
\begin{itemize}
\item $f_0, f_i$~--- выпуклые функции:
\[
f(\alpha \bx_1 + \beta \bx_2) \leq \alpha f(\bx_1) + \beta f(\bx_2),
\]
где $\alpha, \beta \geq 0$ и $\alpha + \beta = 1$.
\item нет аналитического решения
\item существуют эффективные алгоритмы
\item часто сложно <<увидеть>> задачу выпуклой оптимизации
\item существуют приёмы для преобразования задачи к виду~\eqref{eq::conv_opt}
\end{itemize}
\end{frame}

\begin{frame}{Какая задача проще?}

\begin{columns}[T]
\begin{column}{0.5\textwidth}
Поиск независимого под-алфавит максимальной мощности
\begin{equation*}
\begin{split}
&\max\limits_{\bx \in \bbR^n} \sum\limits_{i=1}^n x_i\\
\text{s.t. } & x_i^2 - x_i = 0 \quad i = 1,\ldots, n\\
& x_ix_j = 0 \qquad \forall (i, j) \in \Gamma,
\end{split}
\end{equation*}
где $\Gamma$~--- множество пар
\end{column}
~
\begin{column}{0.5\textwidth}
\small
Truss design
\begin{equation*}
\begin{split}
& \min -2\sum\limits_{i=1}^k\sum\limits_{j=1}^m c_{ij} x_{ij} + x_{00}\\
\text{s.t. } & \sum\limits_{i=1}^k x_i = 1\\
& \lambda_{\min}(\bA) \geq 0,
\end{split}
\end{equation*}
\scriptsize
где $\bA = 
\begin{bmatrix}
x_1 & \ldots & \ldots & \sum\limits_{j=1}^m b_{pj}x_{1j}\\
\vdots & \ddots & & \vdots\\
 & & x_k & \sum\limits_{j=1}^m b_{pj}x_{kj}\\
\sum\limits_{j=1}^m b_{pj}x_{1j} & \ldots & \sum\limits_{j=1}^m b_{pj}x_{kj} & x_{00}
\end{bmatrix}
$
\end{column}
\end{columns}

\end{frame}

\begin{frame}{Почему выпуклость так важна?}
\begin{block}{R. Tyrrell Rockafellar (1935~---)}
The great watershed in optimization is not between linearity and
non-linearity, but convexity and non-convexity.
\end{block}
\begin{itemize}
\item Локальный оптимум является глобальным
\item Необходимое условие оптимальности является достаточным
\end{itemize}
Вопросы:
\begin{itemize}
\item Любую ли задачу выпуклой оптимизации можно эффективно решить?
\item Можно ли эффективно решить невыпуклые задачи оптимизации?
\end{itemize}
\end{frame}

\begin{frame}{Выпуклое множество}
\small
\begin{block}{Выпуклое множество}
Множество $C$ называется выпуклым, если 
\vspace{-4mm}
\[
\forall x_1, \; x_2 \in C, \theta \in [0, 1] \rightarrow \theta x_1 + (1 - \theta)x_2 \in C.
\vspace{-4mm}
\]
$\emptyset$ и $\{ x_0 \}$ также считаются выпуклыми.
\end{block}
Примеры: $\bbR^n$, аффинное множество, луч, отрезок.
\begin{block}{Выпуклая комбинация точек}
Пусть $x_1, \ldots, x_k \in G$, тогда точка $\theta_1 x_1 + \ldots + \theta_k x_k$ при {\color{red}{$\sum\limits_{i=1}^k \theta_i = 1, \;\theta_i \geq 0$}} называется выпуклой комбинацией точек $x_1,\ldots,x_k$.
\end{block}

\begin{block}{Выпуклая оболочка точек}
Множество $\left\{ \sum\limits_{i=1}^k \theta_i x_i \; | \; x_i \in G, {\color{red}{\sum\limits_{i=1}^k \theta_i = 1, \theta_i \geq 0}} \right\}$ называется выпуклой оболочкой множества $G$ и обозначается \textbf{conv}(G).
\end{block}

\end{frame}

\begin{frame}{Операции, сохраняющие выпуклость}
\begin{itemize}
\item Пересечение любого (конечного или бесконечного) числа выпуклых множеств~--- выпуклое множество
\item Образ аффинного отображения выпуклого множества~--- выпуклое множество
\item Линейная комбинация выпуклых множеств~--- выпуклое множество
\item Декартово произведение выпуклых множеств~--- выпуклое множество
\end{itemize}
\end{frame}

\begin{frame}{Примеры}
Проверьте на аффинность и выпуклость следующие множества:
\begin{enumerate}
\item Полупространство: $\{ \bx | \ba^{\T} \bx \leq c \}$
\item Многоугольник: $\{ \bx | \bA\bx \preceq \mathbf{b}, \; \bC\bx = 0 \}$
\item Шар по норме в $\bbR^n$: $B(r, x_c) = \{ x \; | \; \| x - x_c \| \leq r \}$
\item Эллипсоид: $\mathcal{E}(x_c, \bP, r) = \{ x \; | \; (x - x_c)^{\T}\bP^{-1} (x - x_c) \leq r^2 \}$
\item Множество симметричных положительно-определённых матриц: $\bS^n_+ = \{ \bX \in \bbR^{n \times n} \; | \; \bX^{\T} = \bX, \; \bX \succeq 0 \}$
\item $\{ \bX \in \bbR^{n \times n} \; | \; \Tr(\bX) = const \}$
\item Гиперболическое множество: $\{ \bx \in \bbR^n_+ \; | \; \prod\limits_{i=1}^n x_i \geq 1 \}$
\end{enumerate}
\end{frame}

\begin{frame}{Определения функций}
\small
\begin{block}{Выпуклая функция}
Функция $f: X \subset \bbR^n \rightarrow \bbR$ называется выпуклой ({\color{blue}{строго выпуклой}}), если {\color{red}{$X$~--- выпуклое множество}} и для \\ 
$\forall \bx_1, \bx_2 \in X$ и $\alpha \in [0, 1] \; ({\color{blue}{\alpha \in (0, 1)}})$  выполнено:
\vspace{-4mm}
\[
f(\alpha \bx_1 + (1 - \alpha)\bx_2) \leq \; ({\color{blue}{<}}) \; \alpha f(\bx_1) + (1 - \alpha)f(\bx_2)
\]
\end{block}

\begin{block}{Вогнутая функция}
Функция $f$ вогнутая (строго вогнутая), если $-f$ выпуклая (строго выпуклая).
\end{block}

\begin{block}{Сильно выпуклая функция}
Функция $f: X \subset \bbR^n \rightarrow \bbR$ называется сильно  выпуклой с константой $m > 0$, если $X$~--- выпуклое множество и для $\forall \bx_1, \bx_2 \in X$ и $\alpha \in [0, 1]$  выполнено:
\vspace{-4mm}
\[
f(\alpha \bx_1 + (1 - \alpha)\bx_2) \leq \alpha f(\bx_1) + (1 - \alpha)f(\bx_2) - \frac{m}{2} \alpha (1 - \alpha) \| \bx_1 - \bx_2 \|_2^2
\]
\end{block}

\end{frame}

\begin{frame}{Определения множеств}
\footnotesize
\begin{block}{Надграфик (эпиграф)}
Надграфиком функции $f$ называется множество $\text{epi}f = \{ (\bx, y) : \bx \in \bbR^n, \; y \in \bbR, \; y \geq f(\bx) \} \subset \bbR^{n+1}$
\end{block}

\begin{block}{Множество подуровней (множество Лебега)}
Множество подуровня функции $f$ называется следующее множество
\vspace{-4mm}
\[
C_{\gamma} = \{ \bx | f(\bx) \leq \gamma \}.
\]
\end{block}

\begin{block}{Замкнутая функция}
Функция $f$ называется замкнутой, если её надграфик замкнутое множество. 
\end{block}

\begin{block}{Квазивыпуклая функция}
Функция $f$ называется квазивыпуклой, если её область определения и множество подуровней выпуклое множество. 
\end{block}
\end{frame}

\begin{frame}{Критерии выпуклости}
\scriptsize
\begin{block}{Дифференциальный критерий первого порядка}
Функция $f$ выпукла тогда и только тогда, когда определена на выпуклом множестве $X$ и $\forall \bx, \by \in X \subset \bbR^n$ выполнено:
\vspace{-4mm}
\[
f(\by) \geq f(\bx) + \nabla f^{\T}(\bx)(\by - \bx)
\]
\end{block}

\begin{block}{Дифференциальный критерий второго порядка}
Непрерывная и дважды дифференцируемая функция $f$ выпукла тогда и только тогда, когда определена на выпуклом множестве $X$ и $\forall \bx, \by \in \text{ri}(X) \subset \bbR^n$ выполнено:
\vspace{-2mm}
\[
\nabla^2 f(\bx) \succeq 0.
\]
\end{block}

\begin{block}{Связь с надграфиком}
Функция выпукла тогда и только тогда, когда её надграфик выпуклое множество.
\end{block}

\begin{block}{Ограничение на прямую}
Функция $f: X \rightarrow \bbR$ выпукла тогда и только тогда, когда $X$~--- выпуклое множество и выпукла функция $g(t) = f(\bx + t\bv)$ на множестве $\{ t | \bx + t\bv \in X \}$ для всех $\bx \in X$ и $\bv \in \bbR^n$.
\end{block}

\end{frame}

\begin{frame}{Критерии сильной выпуклости}

\begin{block}{Дифференциальный критерий первого порядка}
Функция $f$ сильно выпукла с константой $m$ тогда и только тогда, когда определена на выпуклом множестве $X$ и $\forall \bx, \by \in X \subset \bbR^n$ выполнена:
\vspace{-4mm}
\[
f(\by) \geq f(\bx) + \nabla f^{\T}(\bx)(\by - \bx) + \frac{m}{2}\| \by - \bx \|^2
\]
\end{block}

\begin{block}{Дифференциальный критерий второго порядка}
Непрерывная и дважды дифференцируемая функция $f$ сильно выпукла с константой $m$ тогда и только тогда, когда она определена на выпуклом множестве $X$ и $\forall \bx \in \text{\bf relint}(X) \subset \bbR^n$ выполнено:
\vspace{-2mm}
\[
\nabla^2 f(\bx) \succeq m\bI.
\]
\end{block}
\end{frame}

\begin{frame}{Примеры}
\begin{enumerate}
\item Квадратичная функция: $f(x) = \frac{1}{2}\bx^{\T}\bP\bx + \bq^{\T}\bx + r$, $\bx \in \bbR^n$
\item Нормы в $\bbR^n$
\item $f(\bx) = \log{(e^{x_1} + \ldots + e^{x_n})}$, $\bx \in \bbR^n$
\item Логарифм детерминанта: $f(\bX) = -\log{\det{\bX}}$, $\bX \in \bS^n_{++}$
\item Множество выпуклых функций~--- выпуклый конус
\item Поэлементный максимум: $f(\bx) = \max\{f_1(\bx), f_2(\bx)\}$, dom $f$ = dom $f_1 \; \cap $ dom $f_2$
\item Расширение на бесконечное множество функций: если для $\by \in \calA$ функция $f(\bx, \by)$ выпуклая функция по $\bx$, тогда $\sup\limits_{\by \in \calA} f(\bx, \by) $ выпукла по $\bx$
\item Максимальное собственное значение: $f(\bX) = \lambda_{\max}(\bX) = \sup\{\by^{\T}\bX\by \; | \; \|\by\|_2 = 1\}$ 

\end{enumerate}
\end{frame}

\begin{frame}{Неравенство Йенсена}
 
\begin{block}{Неравенство Йенсена}
Для выпуклой функции $f$ выполнено следующее неравенство:
\vspace{-4mm}
\[
f\left( \sum\limits_{i=1}^n \alpha_i \bx_i \right) \leq \sum\limits_{i=1}^n \alpha_i f(\bx_i),
\vspace{-4mm}
\] 
где $\alpha_i \geq 0$ и $\sum\limits_{i=1}^n \alpha_i = 1$.
\end{block}

или в бесконечномерном случае: $p(x) \geq 0$ и $\int\limits_X p(x) = 1$ 
\vspace{-4mm}
\[
f\left( \int\limits_X p(x)xdx \right) \leq \int\limits_X f(x)p(x)dx
\]
при условии, что интегралы существуют.

\end{frame}

\begin{frame}{Примеры}
\begin{enumerate}
\item Неравенство Гёльдера
\item Неравенство о среднем арифметическом и среднем геометрическом
\item $f(\bE(x)) \leq \bE(f(x))$
\item Выпуклость множества $\{ \bx | \prod\limits_{i=1}^n x_i \geq 1 \}$
\end{enumerate}
\end{frame}

\begin{frame}{Резюме}
\begin{itemize}
\item Организация работы
\item Предмет курса по оптимизации
\item Общие факты об оптимизации
\item Выпуклые множества
\item Выпуклые функции
\end{itemize}
\end{frame}

\end{document}
